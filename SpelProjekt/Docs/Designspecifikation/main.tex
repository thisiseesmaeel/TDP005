\documentclass{TDP005mall}
\usepackage{graphicx}
\graphicspath{ {./images/} }

%\documentclass[a4paper]{article}
\usepackage{fontspec}
\usepackage{tikz}
\usepackage{tikz-uml}
\usetikzlibrary{shapes,arrows}

\usepackage{geometry}
\geometry{a4paper,left=20mm,right=20mm,top=20mm,bottom=30mm}

\newcommand{\version}{Version 1.0}
\author{
  Ismail Safwat, \url{saysa289@student.liu.se}}
\title{Designspecifikation}
\date{2022-11-13}
\rhead{Philip och Ismail}



\begin{document}
\projectpage
\section{Revisionshistorik}
\begin{table}[!h]
\begin{tabularx}{\linewidth}{|l|X|l|}
\hline
Ver. & Revisionsbeskrivning & Datum \\\hline
1.0 & Designspecifikation skapad & 2020-11-25 \\\hline
\end{tabularx}
\end{table}


\section{Detaljbeskrivning}
\subsection{Player}
Player (spelaren) ärvs av Sprite-klassen. Player-klassen har två funktioner: move() och shoot(). move() gör så att spelarobjektet kan röra på sig vänster (←) och höger (→) genom piltangenterna. Sprite-klassen har en funktion som heter wall(), vilket hindrar objekt från att gå bortom spelplanen. shoot() tillåter spelaren att skjuta ett skott, vilket är en klass i sig.

I Sprite-klassen finns en viktig funktion som är relevant för Player: collision(). collision() kommer att kolla om spelaren och en fiende har kolliderat, och om den returnerar true kommer GameOverState att kallas. 

\subsection{Enemy}
Enemy ärvs av Sprite-klassen och har två klassbarn. Enemy-klassen har två funktioner: move(), hit() och killed(). move() gör så att fiendeklassen rör sig genom banan i realtid i ett mönster. Mönstret är att den alltid går från vänster till höger eller vice versa tills den nuddar sidan av skärmen (dvs. wall() anropas av Sprite-klassen), vilket gör att den går ner ett steg och byter riktning och fortsätter så tills den nått spelaren. hit() är en funktion som håller koll hur många gånger en enemy har blivit träffad av ett Shoot-objekt. killed() kollar på hit() och returnerar en bool som säger om fienden ska dö eller inte beroende på vilket värde int life fiendet har. Värdet för life beror om klassbarnet är Enemy1 eller Enemy2. Om fiendet är Enemy1 är life = 1 medan om fiendet är Enemy2 är life = 2. Om killed() returnerar true kommer sprite-klassens funktion remove() att kallas och förstöra Enemy-objektet.

\section{Beskrivning av design}
Designen bygger på två spelstadier: GameState, GameVictoryState och GameOverState. GameState körs så fort användaren öppnar spelet. Om det inte finns några Enemy-objekt kvar i spelet ska GameVictoryState ta över. Om spelaren förlorar (dvs. collision() kallas i Sprite-klassen) kommer GameOverState att ta över och visa en Game Over screen.

Under GameState har vi spelvärlden, vilket består av sprites som representeras av huvudklassen Sprite. Sprite har tre klassbarn: Player, Shoot och Enemy. Alla dessa klassbarn ärver av Sprite. Enemy har i sin tur två klassbarn: Enemy1 och Enemy2.

Fördelen med denna design är att den är mycket enkel och inte komplicerad. En nackdel är förstås att på grund av dess enkelhet finns det mycket som kan utarbetas. Vi skulle kunna lägga till en startmeny och en pausmeny, vilket skulle innebära flera spelstadier. 

\section{Klassdiagram}
\center
\begin{tikzpicture}
  \umlclass[x=0,y=0]{GameState}{
  }{
  }
  \umlclass[x=-6,y=-3.5]{GameVictoryState}{
  }{
  }
  \umlinherit{GameVictoryState}{GameState}
  \umlclass[x=6,y=-3.5]{GameOverState}{
  }{
  }
  \umlinherit{GameOverState}{GameState}
  \umlclass[x=0,y=-3.5]{Sprite}{
  }{
    + wall : void \\
    + remove : void \\
    + collision : bool
  }
  \umlinherit{Sprite}{GameState}
  \umlclass[x=6,y=-20]{Player}{
  }{
    + move : void \\
    + shoot : void
  }
  \umlinherit{Player}{Sprite}
 \umlclass[x=0,y=-7]{Shoot}{
  }{
    + move : void \\
  }
  \umlinherit{Shoot}{Sprite}
  \umlclass[x=-3,y=-17]{Enemy}{
  }{
    + move : void \\
    + hit : int \\
    + killed : bool
  }
  \umlinherit{Enemy}{Sprite}
    \umlclass[x=-6,y=-20]{Enemy1}{
    + life = 1 : int
  }{
  }
    \umlinherit{Enemy1}{Enemy}
    \umlclass[x=0,y=-20]{Enemy2}{
    + life = 2 : int
  }{
  }
    \umlinherit{Enemy2}{Enemy}
    


\end{tikzpicture}
\end{document}


%% game_object
% sprite: variable
% uppdateringsfunktioner (lista/vektor -> iterera igenom -> uppdate() (while))
% kant -> spelplan
% hit() och wall() i kollisionsfunktion/klass i gamestate

% tabell över player i specifkationen
% logik, styrning och utritning
